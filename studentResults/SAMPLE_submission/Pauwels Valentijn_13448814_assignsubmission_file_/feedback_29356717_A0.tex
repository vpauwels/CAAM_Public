\documentclass[a4paper,12pt]{CURSUS}

\usepackage[english]{babel}
\usepackage{latexsym}
\usepackage{amsfonts}
\usepackage{times}
\usepackage{fancyhdr}
\usepackage{rotating}
\usepackage{graphicx}
\usepackage{longtable}
\usepackage{booktabs}
\usepackage[table]{xcolor}

\textwidth 16.5cm
\textheight 23cm
\marginparwidth 0cm
\topmargin 0.54cm
\oddsidemargin 0cm
\evensidemargin 0cm
\parindent 0pt
\parskip 5mm
\headsep 0cm
\footskip 2cm
\flushbottom

\newcommand{\BE}{\begin{equation}}
\newcommand{\EE}{\end{equation}}
\newcommand{\BI}{\begin{itemize}}
\newcommand{\EI}{\end{itemize}}
\newcommand{\BN}{\begin{enumerate}}
\newcommand{\EN}{\end{enumerate}}
\newcommand{\D}{\displaystyle}

\renewcommand {\baselinestretch}{1.2}

\renewcommand{\headrulewidth}{0mm}
\renewcommand{\footrulewidth}{0.2mm}

\pagestyle{fancy}
\addtolength{\headwidth}{4mm}
\renewcommand{\chaptermark}[1]{\markboth{#1}{}}
\fancypagestyle{plain}{
\fancyhead[R,L]{}
\fancyfoot[R]{\thepage}
\fancyfoot[C,L]{}}
\fancyhead[R,L,C]{}
\fancyfoot[R]{\thepage}
\fancyfoot[C,L]{}

\begin{document}

\addtocounter{chapter}{-1}
\renewcommand{\chaptername}{Practice Class Assignment}
\chapter{Deadline August 6, 5 PM}

 Feedback for A0Q0

 The correct distribution type is F.

 The correct H0 hypothesized variable is sigma(A)/sigma(B).

 The correct H0 sign is $>$=.

 The correct H0 hypothesized value is 1

 The correct H1 hypothesized variable is sigma(A)/sigma(B).

 The correct H1 sign is $<$.

 The correct H1 hypothesized value is 1

 The correct test value is 0.6721045369694018

 The correct critical value is 0.46645402121644747

 The conclusion is correct.


 



 -----------------------------


 The question states that we need a flow that is consistently as close as possible to 237.7 ampere, so we are testing for variances.  We need to do an F-test, and the hypotheses are :

 H$_0:\sigma_A/\sigma_B \geq 1$

 H$_1:\sigma_A/\sigma_B < 1$

 We calculate the test value as follows:\begin{equation}F_{calc}=\displaystyle \frac{\sigma_1^2}{\sigma_2^2} = \displaystyle \frac{(9.1)^2}{(11.1)^2}=0.6721\end{equation}

 We have a left-tail test, so we need $\alpha$ of the mass in the left tail.  In Excel (or Open Office Calc) f.inv(0.05,19,21) results in a critical value of 0.4665.  This means that we are not in the rejection region, and we accept the Null Hypothesis.  The conclusion is that the Generator A does not work significantly better than Generator B.

 Mark for this part = 10.0\%

  \noindent\rule{8cm}{0.4pt} 

  \noindent\rule{8cm}{0.4pt} 

 \textbf{Total mark = 10.0}

  \noindent\rule{8cm}{0.4pt} 


\end{document}
